\documentclass{article}
\usepackage{arxiv}

\usepackage[utf8]{inputenc}
\usepackage[english, russian]{babel}
\usepackage[T1]{fontenc}
\usepackage{url}
\usepackage{booktabs}
\usepackage{amsfonts}
\usepackage{nicefrac}
\usepackage{microtype}
\usepackage{lipsum}
\usepackage{graphicx}
\usepackage{natbib}
\usepackage{doi}



\title{Winterstorm risk prediction via machine learning methods}

\author{ Kornilov Nikita\\
	Department of Control and Applied Mathematics\\
	Moscow Institute of Physics and Technology\\
	\texttt{kornilov.nm@phystech.edu} \\
	%% \AND
	%% Coauthor \\
	%% Affiliation \\
	%% Address \\
	%% \texttt{email} \\
	%% \And
	%% Coauthor \\
	%% Affiliation \\
	%% Address \\
	%% \texttt{email} \\
	%% \And
	%% Coauthor \\
	%% Affiliation \\
	%% Address \\
	%% \texttt{email} \\
}
\date{}

\renewcommand{\shorttitle}{\textit{arXiv} Template}

%%% Add PDF metadata to help others organize their library
%%% Once the PDF is generated, you can check the metadata with
%%% $ pdfinfo template.pdf
\hypersetup{
pdftitle={A template for the arxiv style},
pdfsubject={q-bio.NC, q-bio.QM},
pdfauthor={Kornilov Nikita},
pdfkeywords={Maximum risk prediction},
}

\begin{document}
\maketitle

\begin{abstract}
	В статье рассматривается задача прогнозирование рисков суровых снежных бурь по климатическим данным с 1991 года в краткосрочном диапазоне до 20 лет. При этом прогнозирование производится на пространственной сетке большого размера. Для прикладных нужд необходимо достаточно точно прогнозировать экстремальные значения риска, представляющие на практике основной интерес, в то время как в области малых значений временного ряда алгоритм может совершать существенное число ошибок.
\end{abstract}


\keywords{Maximum risk prediction}

\section{Introduction}



\section{Headings: first level}

\subsection{Citations}

\subsection{Figures}


\subsection{Tables}

\subsection{Lists}



\bibliographystyle{unsrtnat}
\bibliography{references}

\end{document}